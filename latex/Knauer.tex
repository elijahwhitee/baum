\documentclass[article, 11pt, oneside, a4paper, english, brazil, sumario=tradicional]{abntex2}
\usepackage{lmodern}
\usepackage[T1]{fontenc}
\usepackage[utf8]{inputenc}
\usepackage{indentfirst}
\usepackage{nomencl}
\usepackage{color}
\usepackage{graphicx}
\usepackage{microtype}
\usepackage{lipsum}
\usepackage[brazilian,hyperpageref]{backref}
\usepackage[alf]{abntex2cite}
\renewcommand{\backrefpagesname}{Citado na(s) página(s):~}
\renewcommand{\backref}{}
\renewcommand*{\backrefalt}[4]{\ifcase #1 Nenhuma citação no texto. \or Citado na página #2. \else Citado #1 vezes nas páginas #2. \fi}
\titulo{Knauer}
\tituloestrangeiro{The importance of the teaching of computation technology in schools}
\autor{
Ícaro Onofre Silva
  %% vinculação corporativa e endereço de contato.''}
}
\local{Campo Limpo Paulista}
\data{24 04 2024}
\definecolor{blue}{RGB}{41,5,195} % alterando o aspecto da cor azul
\makeatletter % informações do PDF
\hypersetup{pdftitle={\@title}, pdfauthor={\@author}, pdfsubject={Modelo de artigo científico com abnTeX2}, pdfcreator={LaTeX with abnTeX2}, pdfkeywords={abnt}{latex}{abntex}{abntex2}{atigo científico}, colorlinks=true, linkcolor=blue, citecolor=blue, filecolor=magenta, urlcolor=blue, bookmarksdepth=4}
\makeatother

\makeindex

\setlrmarginsandblock{3cm}{3cm}{*}
\setulmarginsandblock{3cm}{3cm}{*}
\checkandfixthelayout
\setlength{\parindent}{1.3cm}
\setlength{\parskip}{0.2cm}
\SingleSpacing

\begin{document}

\selectlanguage{brazil}

\frenchspacing

% \twocolumn[    		% Descomente para duas colunas
\maketitle
% resumo em português
\begin{resumoumacoluna}
  Resumo
  \vspace{\onelineskip}
  \noindent
  \textbf{Palavras-chave}: Resumo.
\end{resumoumacoluna}

% resumo em inglês
\renewcommand{\resumoname}{Abstract}
\begin{resumoumacoluna}
  \begin{otherlanguage*}{english}
    Abstract
    \vspace{\onelineskip}

    \noindent
    \textbf{Keywords}: abstract.
  \end{otherlanguage*}
\end{resumoumacoluna}
% ]  				% FIM DE ARTIGO EM DUAS COLUNAS
\begin{center}\smaller

  \textbf{Data de submissão e aprovação}: 24 04 2024.

  %% \textbf{Identificação e disponibilidade}: elemento opcional. Pode ser indicado
  %% o endereço eletrônico, DOI, suportes e outras informações relativas ao acesso.
\end{center}

\newpage

\pdfbookmark[0]{\contentsname}{toc}
\tableofcontents*
\cleardoublepage

\textual

\section{capitulo 1}

Aqui escrevo a tese em si

\postextual


\section*{Agradecimentos}

Nossos mais profundos agradecimentos a todo o corpo docente da Fatec de Jundiaí

\newpage
\bibliography{references}

\end{document}
